%%%%%%%%%%%%%%%%%%%%%%%%%%%%%%%%%%%%%%%%%
% Beamer Presentation
% LaTeX Template
% Version 1.0 (10/11/12)
%
% This template has been downloaded from:
% http://www.LaTeXTemplates.com
%
% License:
% CC BY-NC-SA 3.0 (http://creativecommons.org/licenses/by-nc-sa/3.0/)
%
%%%%%%%%%%%%%%%%%%%%%%%%%%%%%%%%%%%%%%%%%

%----------------------------------------------------------------------------------------
%	PACKAGES AND THEMES
%----------------------------------------------------------------------------------------

\documentclass{beamer}

\mode<presentation> {
	
	% The Beamer class comes with a number of default slide themes
	% which change the colors and layouts of slides. Below this is a list
	% of all the themes, uncomment each in turn to see what they look like.
	
	%\usetheme{default}
	%\usetheme{AnnArbor}
	%\usetheme{Antibes}
	%\usetheme{Bergen}
	%\usetheme{Berkeley}
	%\usetheme{Berlin}
	%\usetheme{Boadilla}
	%\usetheme{CambridgeUS}
	%\usetheme{Copenhagen}
	%\usetheme{Darmstadt}
	%\usetheme{Dresden}
	%\usetheme{Frankfurt}
	%\usetheme{Goettingen}
	%\usetheme{Hannover}
	%\usetheme{Ilmenau}
	%\usetheme{JuanLesPins}
	%\usetheme{Luebeck}
	\usetheme{Madrid}
	%\usetheme{Malmoe}
	%\usetheme{Marburg}
	%\usetheme{Montpellier}
	%\usetheme{PaloAlto}
	%\usetheme{Pittsburgh}
	%\usetheme{Rochester}
	%\usetheme{Singapore}
	%\usetheme{Szeged}
	%\usetheme{Warsaw}
	
	% As well as themes, the Beamer class has a number of color themes
	% for any slide theme. Uncomment each of these in turn to see how it
	% changes the colors of your current slide theme.
	
	%\usecolortheme{albatross}
	%\usecolortheme{beaver}
	%\usecolortheme{beetle}
	%\usecolortheme{crane}
	%\usecolortheme{dolphin}
	%\usecolortheme{dove}
	%\usecolortheme{fly}
	%\usecolortheme{lily}
	%\usecolortheme{orchid}
	%\usecolortheme{rose}
	%\usecolortheme{seagull}
	%\usecolortheme{seahorse}
	%\usecolortheme{whale}
	%\usecolortheme{wolverine}
	
	%\setbeamertemplate{footline} % To remove the footer line in all slides uncomment this line
	%\setbeamertemplate{footline}[page number] % To replace the footer line in all slides with a simple slide count uncomment this line
	
	%\setbeamertemplate{navigation symbols}{} % To remove the navigation symbols from the bottom of all slides uncomment this line
}

\usepackage{graphicx} % Allows including images
\usepackage{booktabs} % Allows the use of \toprule, \midrule and \bottomrule in tables
\usepackage{epigraph}

%----------------------------------------------------------------------------------------
%	TITLE PAGE
%----------------------------------------------------------------------------------------

\title[Intro to probability]{An introduction to probability} % The short title appears at the bottom of every slide, the full title is only on the title page

\author{Ben Lambert} % Your name
\institute[Univ. of Oxford] % Your institution as it will appear on the bottom of every slide, may be shorthand to save space
{
	University of Oxford \\ % Your institution for the title page
	\medskip
	\textit{ben.c.lambert@gmail.com} % Your email address
}
\date{\today} % Date, can be changed to a custom date

\begin{document}
	
	\begin{frame}
		\titlepage % Print the title page as the first slide
	\end{frame}

	\begin{frame}
		\frametitle{Introductions}
		
		Who am I?
		
	\end{frame}

	\begin{frame}
		\frametitle{Course outline}
		
		\begin{itemize}
			\item 9am-10.30am: follow along lecture and problems
			\item 10.30am-10.45am: refreshments break
			\item 10.45am-midday: follow along lecture and problems
		\end{itemize}
		
	\end{frame}

	\begin{frame}
		\frametitle{Resources}
		
		\begin{itemize}
			\item Introduction to probability, Blitzstein and Hwang. Open source book available here: \url{https://projects.iq.harvard.edu/stat110/home}
			\item Seeing theory. A beautiful online resource that has lots of creative ways to think about probability. \url{https://seeing-theory.brown.edu/}
		\end{itemize}
		
	\end{frame}

	\begin{frame}
		\frametitle{Outline}
		\tableofcontents
	\end{frame}

	\section{What is probability and why do we need it?}
	
	\begin{frame}
		\frametitle{What is probability?}
		\epigraph{Mathematics is the logic of certainty; probability is the logic of uncertainty}{Bitzstein and Hwang, 2019}
		
		
	\end{frame}
	
	\begin{frame}
		\frametitle{Why study probability?}
		
		
		
	\end{frame}

	\begin{frame}
		\frametitle{Blitzstein and Hwang's Pebble World}
	\end{frame}

	\section{Probability and counting}
	
	\begin{frame}
		\frametitle{test}
	\end{frame}

	\section{Joint distributions}

	\section{Conditional probability}
	
	\section{Bayes' rule}
	
	\section{Continuous probability distributions}
	
\end{document} 